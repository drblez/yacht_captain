\documentclass[a4paper, 12pt, twoside, final]{scrbook}
\usepackage[warn]{mathtext}
\usepackage[a4paper]{geometry}
\geometry{verbose, tmargin=3cm, bmargin=3cm, lmargin=2cm, rmargin=1.5cm, headheight=1cm, headsep=1cm, footskip=1.5cm}
\usepackage[T2A]{fontenc}
\usepackage[utf8]{inputenc}
\usepackage[russian]{babel}
\usepackage{indentfirst}
\usepackage{misccorr}
\usepackage{epigraph}
%\usepackage{quotchap}
\usepackage{wrapfig}
\usepackage{graphicx}
\usepackage[font=small]{caption}
\usepackage[section]{placeins}
\usepackage{booktabs}
\usepackage[shortcuts, cyremdash]{extdash}
\usepackage{desclist}
\usepackage{appendix}
\usepackage{rotating}
\usepackage{multirow}
\usepackage{longtable}
\usepackage{hyperref}
\usepackage{cmap}
\usepackage[texindy]{imakeidx}
%\usepackage[split]{splitidx}
\makeindex
%\newindex[Алфавитный указатель]{idx}

% висячие и другие строки

\clubpenalty=10000
\widowpenalty=10000

\righthyphenmin=2 % разрешаем переносить два последних символа

\renewcommand{\bfdefault}{b} % плотный жирный

%\renewcommand{\thechapter}{\Asbuk{chapter}}

%\renewcommand\appendixname{Приложение}

%\makeatletter
%\def\redeflsection{\def\l@section{\@dottedtocline{1}{1.5em}{7.8em}}}
%\renewcommand\appendix{\par
%\setcounter{chapter}{0}%
%\setcounter{section}{0}%
%\setcounter{subsection}{0}%
%\def\@chapapp{\appendixname}%
%\def\thechapter{\Asbuk{chapter}}
%\def\thesection{\appendixname\hspace{0.2cm}\@arabic\c@section}}
%\makeatother

\begin{document}

\frontmatter

\author{Составитель: Е.П. Леонтьев}
\title{Школа яхтенного капитана}
\date{1983}
\maketitle

%\newpage{}

\tableofcontents
\listoffigures
\listoftables

\mainmatter

\chapter{Основы теории и устройство крейсерских яхт}
\section{Элементы теории парусной яхты}
\subsection{Требования, предъявляемые к парусной яхте.}

К уровню комфорта и оборудования на борту парусных яхт, в частности крейсерско-гоночных классов, предъявляются известные требования в соответствии с их назначением. Однако самый высокий уровень комфорта, самые совершенные приспособления для настройки парусов, самые современные электронные приборы для управления яхтой окажутся бесполезными, если она не будет обладать мореходными качествами, которые гарантируют безопасность плавания при условиях, определенных районом плавания и назначением яхты.

Яхта должна принимать определенную нагрузку, сохраняя достаточную высоту надводного борта, чтобы не быть залитой на волне. Она должна противостоять давлению ветра на паруса, чтобы не быть опрокинутой внезапно налетевшим шквалом. От яхты требуется хорошая маневренность в тесной гавани, и послушность рулю на штормовой волне. Она должна поддерживать, возможно, более высокую скорость при любых условиях и быть способной идти круто к ветру. Все это и составляет важнейшие мореходные качества: плавучесть, непотопляемость, остойчивость, ходкость, управляемость, поведение при волнении, способность нести паруса. 

Изучение этих качеств является предметом специальной науки - теории корабля. Эта наука определяет также элементы, которые составляют отдельные мореходные качества и которые позволяют оценивать их количественно. Наконец, теория корабля устанавливает связь между формой корпуса судна и характеристиками его мореходных качеств.

В настоящей главе приводятся важнейшие элементы теории корабля в приложении к парусной яхте средних размерений в объеме, необходимом капитану при выходе в плавание.

\end{document}
